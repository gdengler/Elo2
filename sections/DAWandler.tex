\section{DA Wandler\hartl{455}}
  Es gibt 3 Verfahren: Parallelverfahren, Wägeverfahren und Zählverfahren.

\subsection{Parallelverfahren}

\renewcommand{\arraystretch}{1}
\begin{tabular}{|>{\bfseries}p{3cm}|c|p{6.6cm}|}
	\hline
	Strom-DAC \hartl{456} 
	& \includegraphics[width=7cm, valign=t]{pictures/Strom-DAC}
	& {\begin{align*}
      K &=2^N-1 \qquad I_{out} = D \cdot I = D \cdot \frac{V_{Ref}}{R}\\
      I &=\frac{V_{Ref}}{R} \qquad \text{(von einer Quelle)}
	  \end{align*}}
	  \begin{tabular}{lp{5cm}}
	  	K: & Anzahl Stromquellen \\
      D: & Eingangswert (Anzahl Schalter die aktiv sind.) \\
      \multicolumn{2}{l}{Schaltereigenschaften:}\\
      On: & kein Spannungsabfall  \\
      Off: & kein Strom
    \end{tabular} 
	\\ \hline
	String DAC (Voltage Scaling) \hartl{459}
	& \includegraphics[width=5cm, valign=t]{pictures/string_DAC}
	& 
		\begin{description}
  		\item[Vorteile: ] garantierte Stetigkeit
  		\item[Nachteile:] benötigt $2^n$ Widerstände und $2^n$ Schalter, n-to-$2^n$ Decoder(linke Variante),
        er darf nicht belastet werden und hat ein grosser Schaltungsaufwand.
	  \end{description} 
    
	  {\begin{align*}
	  	V_{Out_{ideal}}(D) = \frac{D}{2^n}(V_{Refp}-V_{Refn})+V_{Refn}\\
	  	V_{Out_{real}}(D) = V_{Refn}+\left(V_{Refp}-V_{Refn}\right)\cdot\\
	  		\cdot\frac{D \cdot R_{Load}}{2^n \cdot R \cdot D-R \cdot D^2 + 2^n (R_{Load} + R_{Switch}) }\\
	  	DAC_{error}(D)=\frac{V_{Out_{real}}(D)-V_{Out_{ideal}}(D)}{V_{Out_{ideal}}(D)}\\  q = \frac{1}{2^n} \cdot (V_{Refp}-V_{Refn}) \\
	  \end{align*}}
	  
	\\ \hline
	Segmented String DAC \hartl{459}
	& \includegraphics[width=5cm, height = 4cm, valign=t]{pictures/segmented_string_DAC}
	& \begin{description}
  		\item[Vorteile: ] viel weniger Elemente
  		\item[Nachteile:] benötigt Buffer (offset-frei)
	  \end{description}
	\\ \hline
\end{tabular}
\renewcommand{\arraystretch}{\arraystretchOriginal}

%\newpage

\subsection{Wägeverfahren\hartl{461}} 
\begin{longtable}{|p{3cm}|c|p{8.6cm}|}
	\hline
	\textbf{Spannungs"-summierung \hartl{461}}
	& \includegraphics[width=5cm, valign=t]{./pictures/spannungssummierung.png}
	& {\begin{align*}
		V_{Out} &= \frac{B0\cdot 2^0 + B1 \cdot 2^1+ \ldots + B(n-1)\cdot 2^{n-1}}{2n} \\
    & \cdot (V_{Refp}-V_{Refn}) + V_{Refn}
	  \end{align*}}
    
	  \begin{description}
  		\item[Vorteile:] n Widerstände, n Schalter
  		\item[Nachteile:] nicht garantiert stetig, grosse Wertebereiche für Widerstände, rechnen mit Leitwerten ($G_0 = \frac{1}{8R}$)
	  \end{description}
	\\ \hline
	\textbf{Wägeverfahren mit Ausgangstreiber} \newline
  (Summation gewichteter Ströme)
	& \includegraphics[width=6cm, valign=t]{./pictures/praktisch.png}
	& \[ Idac_{max}=\frac{V_{Refp}-V_{Refn}}{R} \cdot \frac{2^n-1}{2^n} \] \newline
    Vout und Idac\_inv sind differentiel zu einander, dadurch fliesst immer der gleiche Strom
    und der Offsetfehler bleibt konstant.
	\\ \hline
	\textbf{R-2R-Netzwerk \hartl{462}}
	& \includegraphics[width=6cm, valign=t]{./pictures/r2rnetzwerk.png}
	& {
	\begin{align*}
		V_{out,max} &= V_{Refn} \\
		V_{out,min} &= V_{Refn} - R \cdot I_{max}\\
		I_{max}		&= (V_{Refp} - V_{Refn}) (\frac{1}{2R} + \frac{1}{2}\frac{1}{2R} + \frac{1}{4} \frac{1}{2R} + \ldots) \\
					&= (V_{Refp} - V_{Refn})\frac{1}{2R}(2-2^{1-n})
	\end{align*}}
  
  \begin{description}
    \item[Vorteil:] nur 2 unterschiedliche R
    \item[Nachteil:] es muss immer ein Strom fliessen
  \end{description}
  
	\\ \hline
	\textbf{Kapazitiver DAC (Charge Scaling)}
	& \includegraphics[width=6cm, valign=t]{./pictures/kapazitiverDAC.png}
	& { {\bf Vorteil:} Man kann bis zur angelegten Referenzspannung durchschalten \newline
        \begin{align*}
		C1	&= B3 \cdot 8C+B2 \cdot 4C+ B1 \cdot 2C+B0 \cdot C \\
		C2	&= !B3 \cdot 8C+!B2 \cdot 4C+ !B1 \cdot 2C+!B0 \cdot C+C \\
		V_{Out}& =\frac{C1}{C1+C2}\cdot (V_{Refp}-V_{Refn}) + V_{Refn}\\
		\text{mit } & C1+C2=2^n\cdot C
	  \end{align*}}
	\\ \hline
    \textbf{Kapazitiver DAC mit Ausgangstreiber}
    & \includegraphics[width=6cm, valign=t]{./pictures/kapazitiverDACmitAmp.png}
    & {Wenn z.B. B3 auf $V_{Ref1}$ geschaltet wird:\newline
      \begin{align*}
          Q_{C_x} = C_x \cdot \Delta U &= C_x \cdot (V_{Ref1}-V_{Ref2})\\
          \Delta Q_x = \Delta Q_{fb} &= C_x \cdot \Delta V_{Ref}\\
          \Rightarrow \Delta V_{Out} &= \frac{-\Delta Q_{fb}}{C_{fb}} + V_{Ref2} = \Delta V_{Ref} \cdot \frac{-C_x}{C_{fb}} + V_{Ref2} \\ 
          C_x &= B_0 \cdot C_0 + B_1 \cdot C_1 + .... + B_{n-1} \cdot C_{n-1}\\
          V_{out} &= V_{Ref2}-\Delta V_{Out}
      \end{align*}}
    \\ \hline
\end{longtable}

%\newpage

\subsection{Zählverfahren(PWM)\hartl{466}}
\begin{longtable}{|>{\bfseries}p{4cm}|l|p{8cm}|}
	\hline 
	Grundprinzip \hartl{466}
	& \includegraphics[width=6cm, valign=t]{./pictures/pwm_DAC.png}
	& $ V_{Out}=\frac{D}{2^n} \cdot (V_{Refp}-V_{Refn})+V_{Refn} $ \newline
	$V_{out}(t)= V_{out}0 \cdot e^{\frac{-(t-TO)}{R \cdot C}}$\newline
	\begin{tabular}{lp{5cm}}
    \textbf{Vorteile:} 
      &-einfache Schaltung \\
      &-ermöglicht hohe Auflösung \\
      &-Funktioniert ohne analoge Schaltungen on Chip \\
    
    \textbf{Nachteile:}
      &-sehr langsam \\
      &-benötigt grosse Zeitkonstanten 
  \end{tabular}
	\\ \hline
	PWM-Ansteuerung \hartl{466}
	& \parbox[c][2cm]{6cm}{\input{tikz/DAWandler/PWMAnsteuerung.tex}}
	& $\bar{V_{Out}}=\frac{n}{N}V_{Ref}$
	  \begin{tabular}{ll}
		N:&Takte\\
		n:&digitale Eingangsgrösse
	  \end{tabular}
	\\ \hline
\end{longtable}

\newpage

\subsection{Weitere DAC}
\begin{longtable}{|>{\bfseries}p{4cm}|c|p{8cm}|}
	\hline
	Kaskadierte DAC
	& \includegraphics[width=6cm, valign=t]{./pictures/kaskadiertDAC.png}
	& \begin{itemize}
  		\item MS-DAC hat 2 Ausgangsspannungen (Über und unter dem gewünschten
  			$V_{Out}$)
  		\item LS-DAC hat kleine Eingangspannungsdifferenz $\to$ höhere Auflösung der
  			Spannung
	  \end{itemize}
	\\ \hline
	Zyklisch, algorithmischer DAC \hartl{466}
	& \includegraphics[width=6cm, valign=t]{./pictures/zyklischDAC.png}
	& \textbf{Ablauf der Wandlung}
	  \begin{enumerate}
  		\item Die Spannung im S/H löschen (Schalter S1), S3 offen)
  		\item S1 auf den Verstärker-Ausgang schalten
  		\item Laufvariale k wird auf 0 gesetzt
  		\item S2 setzen: VREF oder GND ( abh. $D_{K}$).
  		\item Der Addierer und Verstärker generieren Ausgangssignal
  		\item Im S/H wird die Feedback-Spannung gespeichert (S1)
  		\item X wird um 1 erhöht
  		\item Gehe zu Schritt 4, wenn $X\leq n$
	  \end{enumerate}
	\\ \hline
	Pipelined DAC 
	& \includegraphics[width=6cm, valign=t]{pictures/piplinedDAC}
	& $V_{Out} = (D_0 \cdot 2^{-n} + D_1 \cdot 2^{1-n} + ... + D_{n-1} \cdot 2^{-1})\cdot V_{Ref}$\newline\newline
      Die Latenz beträgt n Zyklen, die Update-Frequenz ist aber n-mal grösser, da die Blöcke n-fach vorliegen.\newline
      LSB ($\mathrm{D_0}$): $\mathrm{V_{Ref}}$ wird n-mal halbiert
	\\ \hline
	Strom-DAC
	& \includegraphics[width=6cm, valign=t]{pictures/stromDAC}
	& \begin{itemize}
  		\item Stromspiegel
  		\item MP0 ist gleich breit wie Stromquellen-MOS $\to \mathrm{I(MP0)=I_{Ref}} \qquad$ MP0: Einheitstransistor
  		\item MP1 ist doppelt so breit wie MP0 $\to \mathrm{I(MP1)}=2*\mathrm{I_{Ref}} \qquad$ MP1: 2 Einheitstransistoren
  		\item MP2 ist doppelt so breit wie MP1 $\to \mathrm{I(MP2)}=4*\mathrm{I_{Ref}} \qquad$ MP2: 4 Einheitstransistoren
  		\item \ldots
	  \end{itemize}
	\\ \hline
\end{longtable}

\subsection{Spezielle Wandler}
\begin{longtable}{|>{\bfseries}p{4cm}|c|p{8cm}|}
  \hline
    Digitales Potentiometer \hartl{460}
    & \includegraphics[width=5cm, valign=t]{pictures/digitales_potentiometer}
    & \begin{itemize}
        \item automatisierter Elektronik-Test möglich
        \item D (digitale Wert) wird im PROM gespeichert
        \item V(A), V(B) können variabel sein
      \end{itemize} \\
  \hline
    Multiplizierende Wandler
    & 
    & Wandler bei denen mit Widerständen aus der Referenzspannung Ströme abgeleitet werden.
    z.B R2R-Netzwerk \\
  \hline
    DAC mit Exponentieller Funktion
    &\includegraphics[width=6cm, valign=t]{pictures/DAC_exp.png}
    & \\
  \hline
\end{longtable}

\subsection{Ausgangsverstärker}
\begin{minipage}{7cm}
	\includegraphics[width=6.5cm]{pictures/Ausgangsverstaerker.png}
\end{minipage}
\begin{minipage}{12cm}
  $C_{Filter}$ dämpft Glitches beim Umschalten des Digitalwertes.
  \begin{align*}
  	V_{opp} &= 0V \hdots I_{OUT} \cdot (25\Omega || 1.5k\Omega) \\
  	V_{opn} &= 0V \hdots \overline{I_{OUT}} \cdot (25\Omega || 1.5k\Omega) \cdot \frac{1k\Omega}{1.5k\Omega} = V_{opp} \cdot \frac{1k\Omega}{1.5k\Omega} \\
  	A_{pos} &= 1 + \frac{1k\Omega}{500\Omega + 25\Omega}
  \end{align*}
\end{minipage}

