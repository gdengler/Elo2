\section{OpAmps AC}

Der Operationsverstärker ist in allgemeiner Näherung ein Tiefpass-Filter n-ter Ordnung mit linearer Verstärkung. \\
\renewcommand{\arraystretch}{1.2}
 \begin{tabular}{|p{0.3\linewidth}|p{0.3\linewidth}|p{0.3\linewidth}|}
 	\hline
 	Frequenzgang allgemein
 		& \large{$A(s) = \frac{A_{0}}{(1+\frac{s}{\omega_{p_1}})(1+\frac{s}{\omega_{p_2}})\dots}$}
 		& $A_{0}=$ Lineare Verstärkung \newline $\omega{p_i}=$ Polkreisfrequenzen \\
 	\hline
 	GBP(gain-bandwidth-product) & $BW = \frac{GBP}{gain}$ & BW = Bandwidth
 	\\ \hline
 	ENB (equivalent noise bandwidth) & $ENB = \frac{\pi}{2} \cdot BW$ \\ \hline
 \end{tabular}

\subsection{Open-Loop/Closed-Loop Verhalten}

\begin{tabular}{|p{0.45\linewidth}|p{0.45\linewidth}|}
	\hline
	\textbf{Open-Loop}
		& \textbf{Closed-Loop}\\
	\hline
	\multicolumn{2}{|c|}{\textbf{Blockschemas}}\\
	\hline
    \vspace{-7mm}
	\begin{center}
	 	\includegraphics[height=2cm, valign=t]{./pictures/opAmpOL.png}
	\end{center}
		& \vspace{-7mm}
          \begin{center}
			\includegraphics[height=2cm, valign=t]{./pictures/opAmpCL.png}
		  \end{center}\\
	\hline
	\multicolumn{2}{|c|}{\textbf{Frequenzgänge}}\\
	\hline
	\large{$A_{ol}(s)=\frac{A_{ol_0}}{(1+\frac{s}{\omega_{p_{ol_1}}})(1+\frac{s}{\omega_{p_{ol_2}}})\dots}$}
	& $\begin{aligned}
        A_{cl}(s) &= \frac{A_{cl_0}}{(1+\frac{s}{\omega_{p_{cl_1}}})(1+\frac{s}{\omega_{p_{cl_2}}})\dots} = \frac{T_{in}\cdot A_{ol_0}}{1+\beta(s)\cdot A_{ol_0}}\\
		\beta(s) &= \frac{V_{opn}}{V_{out}}\;\text{oder}\;\frac{V_{opp}}{V_{out}}\\
		T_{in}(s) &= \frac{V_{opn}}{V_{in}}\;\text{oder}\;\frac{V_{opp}}{V_{in}}\\
        V_{out} &= A_{cl}(s)\cdot V_{in} = \frac{T_{in}(s)\cdot A_{ol}(s)}{1 + \underbrace{A_{ol}(s)\cdot \beta(s)}_{T_s(s):Loop-Gain}} \cdot V_{in}
	   \end{aligned}$\\
	\hline
\end{tabular}
\\ \\
 \renewcommand{\arraystretch}{1.5}
\begin{tabular}{m{0.45\linewidth}m{0.45\linewidth}}
	Durch das Schliessen des Loops wird die Bandbreite vergrössert, das gain-bandwidth-product(GBP) bleibt jedoch konstant. Die Verstärkung wird jedoch um $T_{s0}(s)$ (Linearer Loop Gain) reduziert. Der Phasengang wird durch das Verschieben des ersten Poles auch verändert, wie folgende Grafik zeigt.
	\subsubsection{Open Loop Gain und Pole aus Bodediagramm lesen}
	Der Open Loop Gain lösst sich bei $\omega \Rightarrow 0 $ auslesen\newline 
	Die Pole liegen da, wo die Phase sich um 45\textdegree ändert. \newline
	& \begin{center}
        \includegraphics[width=6.7cm, valign=t]{./pictures/AolAcl.png}
    \end{center}
\end{tabular}
\vspace{-7mm}

\subsection{Stabilität des Systems}
\begin{tabular}{m{0.45\linewidth}m{0.45\linewidth}}
    Um die Stabilität des OpAmps zu betrachten, wird der Loop geöffnet. Damit das System stabil ist, darf das       
    Fehlersignal sich selbst nicht verstärken. Damit dies der Fall ist muss der Loop gain bei einer Phase von $-180^\circ$
    nicht $>1$ sein, da das Vergleichsglied die Phase noch um $180^\circ$ dreht. Ein Mass für die Stabilität ist die 
    Phasenmarge (Phase Margin) und die Verstärkungsmarge (Gain Margin). Optimal ist ein Phase Margin von $60^{\circ}$.
    
    Die UTF ist dann stabil, wenn sie nie $\infty$ wird d.h der Nenner darf nie 0 werden.
    & \begin{center}
        \includegraphics[width=5cm, valign=t]{./pictures/margins.png}
    \end{center}
\end{tabular}

% ----------------------------------------------------------------------------------------------------  
\subsection{OP als Regelkreis}  
\vspace{-1.5\topsep}
\begin{longtable}[t]{|p{5cm}|p{12.7cm}|}
    \hline  
    \multicolumn{2}{|l|}{\bf Nicht-invertierender OP}
    \\ \hdashline
    \includegraphics[width=5cm, valign=t]{pictures/opAmpNI.png}\newline\newline
    \includegraphics[width=5cm]{pictures/OPnichtInv.png}
    & {\vspace{-1.5\topsep}
        \begin{align*}
            A &= A_{ol}\\
            T_{in} &= 1\\
            \beta &= \frac{R_1}{R_1 + R_f}\\
            \frac{V_{out}}{V_{in}} &= T = \frac{T_{in}}{\beta}\cdot \frac{1}{1+ \frac{1}{\beta 
            \cdot A_{ol}}} = \frac{T_{in}\cdot A_{ol}}{1 + A_{ol}\cdot \beta} = 
            \frac{A_{ol}}{1 + A_{ol} \cdot \frac{R_1}{R_1 + R_f}} \approx \frac{R_1 + R_f}{R_1} =\frac{1}{\beta}
        \end{align*}
    }
    \\ \hline
% ---------------------------------------------------------------------------------------------------- 
    \multicolumn{2}{|l|}{\bf Invertierender OP}
    \\ \hdashline
    \includegraphics[width=5cm, valign=t]{pictures/opAmpInv.png}\newline\newline
    \includegraphics[width=5cm]{pictures/OPInv.png}
    & {\vspace{-1.5\topsep}
        \begin{align*}
            A &= A_{ol}\\
            T_{in} &= \frac{R_f}{R_1 + R_f} = 1 - \beta\\
            \beta &= \frac{R_1}{R_1 + R_f}\\
            \frac{V_{out}}{V_{in}} &= T = -\frac{T_{in}}{\beta}\cdot 
            \frac{1}{1+ \frac{1}{\beta \cdot A_{ol}}} =
            \frac{-T_{in} \cdot A_{ol}}{1+A_{ol} \cdot \beta}\approx -\frac{R_f}{R_1} = -\frac{1-\beta}{\beta}
        \end{align*}
    }
    \\ \hline
\end{longtable}
% ---------------------------------------------------------------------------------------------------- 
\vspace{-2.5\topsep}
\begin{longtable}[t]{|p{5cm}|p{12.7cm}|}
    \hline  
    \multicolumn{2}{|l|}{\bf Closed-Loop-Frequenzgang}
    \\ \hdashline
    \includegraphics[width=5cm, valign=t]{pictures/ClosedLoopFreqGang.png}
    & {\vspace{-1.5\topsep}
        \begin{align*}
            \intertext{Frequenzabhängige Verstärkung Opamp:}
            A_{ol}(s) &= \frac{A_0}{1 + s/\omega_p} = \frac{A_0}{1+\frac{A_0}{2\pi \cdot GBP} \cdot s}\\
            \intertext{eingesetzt in folgende Formel:}
            V_{out} &= \frac{A_{ol}}{1 + \beta \cdot A_{ol}} \cdot V_{in}\\
            \intertext{ergibt:}
            A_{cl}(s) &= \frac{\frac{A_0}{1 + s/\omega_p}}{1 + \beta \frac{A_0}{1 + s/\omega_p}} = 
            \frac{A_0}{(1 + A_0 \beta) + s/\omega_p} = 
            \frac{\frac{A_0}{1 + A_0 \beta}}{1 + s \frac{1}
            {\omega_p(1 + A_0 \beta)}} = 
            \frac{A_{0_{cl}}}{1 + s\frac{1}{\omega_{p_{cl}}}}
        \end{align*}
        \newline
        Für $\mathrm{A_0 \cdot \beta \gg 1}$ gilt:\newline
        \vspace{-1.5\topsep}
        \begin{itemize}[leftmargin=*]
            \item Verstärkung $\mathrm{A_{0cl} = 1/\beta}$
            \item Bandbreite wird vergrössert um den Faktor $\mathrm{A_0 \cdot\beta}$
            \newline
        \end{itemize}
        \begin{tabular}{lp{8cm}}
          $\left.\begin{matrix}
            \mathrm{GBW = A_0 \cdot \omega_p}\\
            \mathrm{GBP = \dfrac{A_0 \cdot \omega_p}{2\pi}}
          \end{matrix}\right\rbrace$ &
          Frequenz bei welcher der Amplitudengang der Verlängerung des 1.Pols die 0dB-Linie schneidet.
        \end{tabular} \newline
        Closed-Loop-Bandbreite: Schnittpunkt $\mathrm{A_{cl}}$ mit $\mathrm{A(\omega)}$\newline
        Loop Gain: $A_{ol}(s) \cdot \beta = A(s) \cdot \beta$ oder $A_0 - A_{0cl}$ (nach Grafik) \newline
        Unity-Gain: Frequenz wo der Amplitudengang (aller Pole) die 0dB-Linie schneidet.
    }
    \\ \hline
\end{longtable}


\subsection{DC-Betrachtung: (endliche Verstärkung $\mathrm{A_{ol}}$)}
\begin{itemize}
    \item Eingangsdifferenzspannung: $\mathrm{\frac{V_{out}}{A_{ol}}}$
    \item Verstärkungsfehler: $\mathrm{\sim\frac{1}{\beta \cdot A_{ol}}}$
    \item Eingangsimpedanz vergrössert um $\mathrm{\beta \cdot A_{ol}}$
    \item Ausgangsimpedanz dividiert durch $\mathrm{\beta \cdot A_{ol}}$
  \end{itemize}  
